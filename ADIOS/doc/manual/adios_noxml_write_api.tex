\chapter{ADIOS No-XML Write API }
\label{chapter-noxml-api}

ADIOS provides an option of writing data without loading an XML configuration file. 
This set of APIs is designed to cater to output data , which is not definable from 
the start of the simulation; such as an adaptive code. Using the no-XML API allows 
users to change their IO setup at runtime in a dynamic fashion.  This section discusses 
the details of no-XML write API's and demonstrates how they can be used in a program. 

\section{No-XML Write API Description}

This section lists routines that are needed for ADIOS no-XML functionalities. These 
routines prepare ADIOS metadata construction, for example, setting up groups, variables, 
attributes and IO transport method, and hence must be called before any other ADIOS 
I/O operations, i.e., adios\_open, adios\_group\_size, adios\_write, adios\_close. 
A common practice of using no-XML API is to first initialize ADIOS by calling 
adios\_init\_noxml 
and call adios\_allocate\_buffer to allocate the necessary buffer for ADIOS to achieve 
best performance. Subsequently, declare a group via adios\_declare\_group, and then 
adios\_define\_var needs to be repetitively called to define every variable 
for the group.  In the end, adios\_select\_method needs to be called to choose 
a specific transport method.

\textbf{adios\_init\_noxml ---} initialize no-XML ADIOS

\textbf{adios\_set\_max\_buffer\_size ---} specify maximum size for ADIOS buffer in MB

\textbf{adios\_declare\_group ---} declare an ADIOS group 

\textbf{adios\_define\_var ---} define an ADIOS variable for an ADIOS group

\textbf{adios\_define\_attribute ---} define an ADIOS attribute passed in a string for an ADIOS group

\textbf{adios\_define\_attribute\_byvalue ---} define an ADIOS attribute by value for an ADIOS group

\textbf{adios\_write\_byid ---} write a variable, identified by the ID returned by adios\_define\_var, instead of by name

\textbf{adios\_select\_method ---} associate an ADIOS transport method, such as MPI, 
POSIX method with a particular ADIOS group. The transport methods that are supported 
can be found in Chapter \ref{chapter-methods}.

\subsection{adios\_init\_noxml}

As opposed to adios\_init(), adios\_init\_noxml initializes ADIOS without loading
and XML configuration file. Note that adios\_init\_noxml is required to be called only 
once and before any other ADIOS calls. 

\begin{lstlisting}[alsolanguage=C,caption={},label={}]
int adios_init_noxml (MPI_Comm comm)
\end{lstlisting}

Input: 
\begin{itemize}
\item MPI communicator. All processes that uses ADIOS for writing data must be included in the group of this communicator.
\end{itemize}

Fortran example: 
\begin{lstlisting}[alsolanguage=Fortran,caption={},label={}]
call adios_init_noxml (comm, ierr)
\end{lstlisting}

\subsection{adios\_set\_max\_buffer\_size}
This routine sets a maximum on the buffer size used by ADIOS to buffer data in one adios\_open()...adios\_close() operation.
If multiple operations are going on at the same time, each buffer will be limited independently.

\begin{lstlisting}[alsolanguage=C,caption={},label={}]
void adios_set_max_buffer_size (uint64_t buffer_size)
\end{lstlisting}

Fortran example:
\begin{lstlisting}[alsolanguage=Fortran,caption={},label={}]
call adios_set_max_buffer_size (sizeMB)
\end{lstlisting}


%\subsection{adios\_allocate\_buffer}
%
%The adios\_allocate\_buffer routine allocates a memory buffer for ADIOS to buffer all writes before writing all data at once. 
%
%\begin{lstlisting}[alsolanguage=C,caption={},label={}]
%int adios_allocate_buffer (
%    enum ADIOS_BUFFER_ALLOC_WHEN adios_buffer_alloc_when,
%    uint64_t buffer_size)
%\end{lstlisting}
%
%Input: 
%\begin{itemize}
%\item adios\_buffer\_alloc\_when - indicates when ADIOS buffer should be allocated. 
%The value can be either {\small ADIOS\_BUFFER\_ALLOC\_NOW} or 
%{\small ADIOS\_BUFFER\_ALLOC\_LATER}. See Section \ref{section-xml-buffers-pecification} 
%for more details on ADIOS buffer.
%\item buffer\_size - the size of ADIOS buffer in MB. 
%\end{itemize}
%
%
%Fortran example: 
%\begin{lstlisting}[alsolanguage=Fortran,caption={},label={}]
%call adios_allocate_buffer (sizeMB, ierr)
%\end{lstlisting}
%
%Note that, as opposed to the C function, the Fortran subroutine doesn't have 
%adios\_buffer\_alloc\_when argument as it supports only the 
%{\small ADIOS\_BUFFER\_ALLOC\_NOW} option.
%

\subsection{adios\_declare\_group}

This function is used to declare a new ADIOS group. The concept of ADIOS group, variable, 
attribute is detailed in Chapter \ref{chapter-xml}.

\begin{lstlisting}[alsolanguage=C,caption={},label={}]
int adios_declare_group (int64_t * id, 
                         const char * name,
                         const char * time_index,
                         enum ADIOS_FLAG stats)
\end{lstlisting}

Input: 

\begin{itemize}
\item name - string containing the annotation name of the group 

\item time\_index - string containing the name of time attribute. If there is no time 
attribute, an empty string (\verb+""+) should be passed

\item stats - a flag indicating whether or not to generate ADIOS statistics during writing, 
such as min/max/standard deviation. The value of \textit{stats} can be either 
\verb+adios_flag_yes+ or \verb+adios_flag_no+. If stats is set to \verb+adios_flag_yes+, 
ADIOS internally calculates and outputs statistics for each processor automatically. 
The downside of turning stats on is that it consumes more CPU and memory during writing 
and the metadata will be larger. 
\end{itemize}

Output: 
\begin{itemize}
\item id - pointer to the ADIOS group structure
\end{itemize}

Fortran example: 
\begin{lstlisting}[alsolanguage=Fortran,caption={},label={}]
call adios_declare_group (m_adios_group, "restart", "iter", 1, ierr)
\end{lstlisting}

\subsection{adios\_define\_var}

This API is used to declare an ADIOS variable for a particular group. In previous versions, the \verb+name+ was used to denote the base name part of a full path. It could be used in the past to identify the variable in the function calls. Therefore, a separate \verb+path+ argument is provided to define the path for the variable. Since version 1.6, write and read calls must match the full path (\verb+<path>/<name>+) so it's easier to pass the full path in the \verb+name+ argument and leave the \verb+path+ argument empty or NULL. Nevertheless, the old way of doing this is still supported.  

\begin{lstlisting}[alsolanguage=C,caption={},label={}]
int64_t adios_define_var (int64_t group_id, 
                          const char * name,
                          const char * path,
                          enum ADIOS_DATATYPES type,
                          const char * dimensions,
                          const char * global_dimensions,
                          const char * local_offsets)
\end{lstlisting}

Input: 
\begin{itemize}
\item group\_id - pointer to the internal group structure (returned by adios\_declare\_group 
call)

\item name - string containing the name part of a variable (can be the full path)

\item path - string containing the path of an variable (deprecated)

\item type - variable type (e.g., adios\_integer or adios\_double) 

\item dimensions - string containing variable local dimension. 
If the variable is a scalar, an empty string (\verb+""+) is expected. 
See Section \ref{section-xml-variables} for details on variable local dimensions.

\item global\_dimensions - string containing variable global dimension. If the variable 
is a scalar or local array, an empty string (\verb+""+) is expected. 
See Section \ref{section-xml-global-arrays} for details on global dimensions.

\item local\_offsets - string containing variable local offset. If the variable is a 
scalar or local array, an empty string (\verb+""+) is expected.
\end{itemize}

Return value:

A 64bit ID of the definition that can be used when writing multiple sub-blocks 
of the same variable within one process within one output step. 

Fortran example: 
\begin{lstlisting}[alsolanguage=Fortran,caption={},label={}]
call adios_define_var (m_adios_group, "temperature", "", 6, &
                       "NX", "G", "O", varid)
\end{lstlisting}

\subsection{adios\_set\_transform}

\begin{lstlisting}[alsolanguage=C,caption={},label={}]
int adios_set_transform (int64_t var_id, const char *transform_type_str)
\end{lstlisting}

Input:
\begin{itemize}
\item id---id returned by the corresponding adios\_define\_var() call
\item transform\_type\_str---string of selected transform method; use the same string as in the XML transform attribute of the <var> element.
\end{itemize}

Return value = adios\_errno. 0 indicates success, otherwise adios\_errno is set and the same value is returned. 

Fortran example: 
\begin{lstlisting}[alsolanguage=Fortran,caption={},label={}]
call adios_set_transform (var_id, "zlib", ierr)
\end{lstlisting}


\subsection{adios\_write\_byid}
\verb+adios\_write()+ finds the definition of a variable by its name. If you write
a variable multiple times in an output step, you must define it as many times as you
write it and use the returned IDs in \verb+adios_write_byid()+ to identify what you
are writing.

\begin{lstlisting}[alsolanguage=C,caption={},label={}]
int adios_write_byid (int64_t fd_p, int64_t id, void * var)
\end{lstlisting}

Input:
\begin{itemize}
\item fd\_p---pointer to the internal file structure
\item id---id returned by the corresponding adios\_define\_var() call
\item var ---the address of the data element defined need to be written
\end{itemize}

Fortran example: 
\begin{lstlisting}[alsolanguage=Fortran,caption={},label={}]
call adios_write_byid (handle, id, v, ierr)
\end{lstlisting}


\subsection{adios\_define\_attribute}

This API is used to declare an ADIOS attribute for a particular group. The value is passed in a string, which is converted to the desired type inside the function. It is not appropriate to define a floating-point type attribute with high-precision, and cannot be used to define an array attribute. For those cases, see the next function below, 
adios\_define\_attribute\_byvalue. 

This function is the equivalent function for the XML-based attribute definition, where the values are represented as strings in the XML document. See section \ref{chapter-xml-attributes} for more detais on ADIOS attribute.

\begin{lstlisting}[alsolanguage=C,caption={},label={}]
int adios_define_attribute (int64_t group,
                            const char * name, 
                            const char * path,
                            enum ADIOS_DATATYPES type,
                            const char * value,
                            const char * var)
\end{lstlisting}

Input:
\begin{itemize}
\item group - pointer to the internal group structure (returned by adios\_declare\_group)

\item name - string containing the annotation name of an attribute

\item path - string containing the path of an attribute

\item type  - type of an attribute

\item value - pointer to a memory buffer that contains the value of the attribute

\item var - name of the variable which contains the attribute value. This argument needs 
to be set if argument \verb+value+ is null.  
\end{itemize}

Output:
\begin{itemize}
\item None
\end{itemize}

Fortran example: 
\begin{lstlisting}[alsolanguage=Fortran,caption={},label={}]
call adios_define_attribute (m_adios_group, "date", "", 9, &
                             "Feb 2010", "" , ierr)
\end{lstlisting}


\subsection{adios\_define\_attribute\_byvalue}
\label{section-define-attribute-byvalue}

This API is used to declare an ADIOS attribute by passing the exact value for the desired type in a void* pointer. It also allows to define an array attribute of \verb+nelems+ number of elements. The pointer should point to the array of values. 

\begin{lstlisting}[alsolanguage=C,caption={},label={}]
int adios_define_attribute_byvalue (int64_t group,
                                    const char * name, 
                                    const char * path,
                                    enum ADIOS_DATATYPES type,
                                    int nelems,
                                    void * values)
\end{lstlisting}

Input:
\begin{itemize}
\item group - pointer to the internal group structure (returned by adios\_declare\_group)

\item name - string containing the annotation name of an attribute

\item path - string containing the path of an attribute

\item type  - type of the attribute

\item nelems - number of elements in the array of attributes 

\item values - pointer to a memory buffer that contains the value(s) of the attribute
 
\end{itemize}

Output:
\begin{itemize}
\item None
\end{itemize}

Simple string attributes should be defined with the type \verb+adios_string+, nelems=1 and values should be the \verb+char*+ type pointer. For an attribute of an array of strings, use the type \verb+adios_string_array+, and values should be a pointer of \verb+char**+ type, i.e. values[i] is a string, where $0<=i<nelems$. The length of the individual strings can be different in C.

Fortran example: 

Note that string attributes will loose the trailing space characters in the output. 

\begin{lstlisting}[alsolanguage=Fortran,caption={},label={}]
    integer, dimension(5)   :: someints = (/ 5,4,3,2,1 /)
    real*8, dimension(5)    :: somedoubles = (/ 5.55555, 4.4444, 3.333, 2.22, 1.1 /)
    character(len=5), dimension(3) :: three_strings = (/ "X    ", "Yy   ", "ZzZ  " /)

    call adios_define_attribute_byvalue (m_adios_group, &
            "single_string","", 1, "A single string attribute", adios_err)
    call adios_define_attribute_byvalue (m_adios_group, &
            "three_strings","", 3, three_strings, adios_err)
    call adios_define_attribute_byvalue (m_adios_group, &
            "single_double","", 1, somedoubles, adios_err)
    call adios_define_attribute_byvalue (m_adios_group, &
            "five_ints",    "", 5, someints, adios_err)
\end{lstlisting}


\subsection{adios\_select\_method}
This API is used to choose an ADIOS transport method for a particular group. 

\begin{lstlisting}[alsolanguage=C,caption={},label={}]
int adios_select_method (int64_t group, 
                         const char * method,
                         const char * parameters,
                         const char * base_path)
\end{lstlisting}

Input:
\begin{itemize}
\item group - pointer to the internal group structure (returned by adios\_declare\_group 
call)

\item method - string containing the name of transport method that will be invoked during 
ADIOS write. The list of currently supported ADIOS methods can be found in Chapter 
\ref{chapter-methods}.

\item parameters - string containing user defined parameters that are fed into transport 
method.  For example, in MPI\_AMR method, the number of subfiles to write can be 
set via this argument (see section \ref{section-method-mpiamr}).
This argument will be ignored silently if a transport method doesn't support 
the given parameters.

\item base\_path -  string specifing the root directory to use when writing to disk. 
By default, methods open files with relative paths relative to the current directory, 
but base\_path can be used to change this behavior.
\end{itemize}

Fortran example: 
\begin{lstlisting}[alsolanguage=Fortran,caption={},label={}]
call adios_select_method (m_adios_group, "MPI", "", "", ierr)
\end{lstlisting}

\section{Create a no-XML ADIOS program}

Below is a programming example that illustrates how to write a double-precision 
array t and a double-precision array with size of NX using no-XML API.
A more advanced example on writing out data sub-blocks is listed in the 
appendix Section \ref{section-appendix-writing-subblocks}. 

\begin{lstlisting}[alsolanguage=Fortran,caption={ADIOS no-XML example},label={}]
program adios_global 
    use adios_write_mod
    implicit none
    include "mpif.h"
    character(len=256) :: filename = "adios_global_no_xml.bp" 
    integer :: rank, size, i, ierr
    integer,parameter :: NX=10
    integer :: O, G    
    real*8, dimension(NX) :: t 
    integer :: comm
    integer :: ierr
    integer*8 :: adios_groupsize, adios_totalsize
    integer*8 :: adios_handle
    integer*8 :: m_adios_group
    integer*8 :: varid ! dummy variable definition ID

    call MPI_Init (ierr)
    call MPI_Comm_dup (MPI_COMM_WORLD, comm, ierr)
    call MPI_Comm_rank (comm, rank, ierr) 
    call MPI_Comm_size (comm, size, ierr)
    call adios_init_noxml (comm, ierr)
    call adios_set_max_buffer_size (10)
    call adios_declare_group (m_adios_group, "restart", "iter", 1, ierr) 
    call adios_select_method (m_adios_group, "MPI", "", "", ierr)
    
    !
    ! Define output variables
    !

    ! define integer scalars for dimensions and offsets
    call adios_define_var (m_adios_group, "NX","", 2, &
                           "", "", "", varid) 
    call adios_define_var (m_adios_group, "G", "", 2 &
                           "", "", "", varid) 
    call adios_define_var (m_adios_group, "O", "", 2 &
                           "", "", "", varid)
        
    ! define a global array
    call adios_define_var (m_adios_group, "temperature", "", 6 &
                           "NX", "G", "O", varid)

    !
    ! Write data 
    !
    call adios_open (adios_handle, "restart", filename, "w", comm, ierr)
    
    adios_groupsize = 4 + 4 + 4 + NX * 8
    call adios_group_size (adios_handle, adios_groupsize, &
                           adios_totalsize, ierr)
    G = NX * size 
    O = NX * rank 
    do i = 1, NX
        t(i) = rank * NX + i - 1 
    enddo
    
    call adios_write (adios_handle, "NX", NX, ierr)
    call adios_write (adios_handle, "G", G, ierr)
    call adios_write (adios_handle, "O", O, ierr)
    call adios_write (adios_handle, "temperature", t, ierr)
    call adios_close (adios_handle, ierr) 

    call MPI_Barrier (comm, ierr)
    call adios_finalize (rank, ierr)
    call MPI_Finalize (ierr) 
end program
\end{lstlisting}


\section{No-XML Write API for visualization schema Description}
This section lists routines that are needed for ADIOS no-XML functionalities
that provide support for the visualization schema. These routines prepare ADIOS
attributes that is consistant for different kinds of meshes and could be
understood by both scientists and visulization experts. These attributes will
be used by ADIOS read API and visualization tool will be able to reconstruct the
mesh from the attributes stored in ADIOS files.

\textbf{adios\_define\_schema\_version ---} Defines the schema version

\textbf{adios\_define\_var\_mesh ---} Assigns a mesh to a variable

\textbf{adios\_define\_var\_centering ---} Defines the variable centering on the mesh

\textbf{adios\_define\_var\_timesteps ---} Defines the variable time steps

\textbf{adios\_define\_var\_timescale ---} Define the variable time scale

\textbf{adios\_define\_var\_timeseriesformat ---} Defines the variable time series format or padding pattern for images

\textbf{adios\_define\_var\_hyperslab ---} Defines a variable hyper slab (sub set or super set)

\textbf{adios\_define\_mesh\_timevarying ---} Indicate the mesh will change with time or not

\textbf{adios\_define\_mesh\_timesteps ---} Define the time steps at the mesh level (for all variables on that mesh)

\textbf{adios\_define\_mesh\_timescale ---} Define the time scale at the mesh level (for all variables on that mesh)

\textbf{adios\_define\_mesh\_timeseriesformat ---} Define the time series formatting at the mesh level (for all variables on that mesh)

\textbf{adios\_define\_mesh\_group ---} Indicates where (which ADIOS group) mesh variables are stored

\textbf{adios\_define\_mesh\_file ---} Define a external file where mesh variables are written

\textbf{adios\_define\_mesh\_uniform ---} Defines a uniform mesh

\textbf{adios\_define\_mesh\_rectilinear ---} Defines a rectilinear mesh

\textbf{adios\_define\_mesh\_structured ---} Defines a structured mesh

\textbf{adios\_define\_mesh\_unstructured ---} Defines a unstructured mesh

\subsection{adios\_define\_schema\_version}
This function defines which version of schema is used for visualization in ADIOS.

\begin{lstlisting}[alsolanguage=C,caption={},label={}]
int adios_define_schema_version (int64_t group_id, char * schema_version)
\end{lstlisting}

Input:
\begin{itemize}
\item group\_id - id of the internal group structure (returned by adios\_declare\_group call)
\item schema\_version - string containing the version of schema
\end{itemize}

Fortran example:
\begin{lstlisting}[alsolanguage=Fortran,caption={},label={}]
call adios_define_schema_version (m_adios_group, "1.1")
\end{lstlisting}


\subsection{adios\_define\_var\_mesh}
This API assigns a mesh to a variable.

\begin{lstlisting}[alsolanguage=C,caption={},label={}]
int adios_define_var_mesh (int64_t group_id, 
                           const char * varname, 
                           const char * meshname)
\end{lstlisting}

Input:
\begin{itemize}
\item group\_id - id the internal group structure (returned by adios\_declare\_group call)
\item varname - string containing the variable name which is going to be visualized
\item meshname - string containing the mesh name which is used to visualize the variable
\end{itemize}

Fortran example:
\begin{lstlisting}[alsolanguage=Fortran,caption={},label={}]
adios_define_var_mesh (m_adios_group, "NX", "uniformmesh")
\end{lstlisting}


\subsection{adios\_define\_var\_centering}
This API efines the variable centering on the mesh. Variables could be cell centered or point centered.

\begin{lstlisting}[alsolanguage=C,caption={},label={}]
int adios_define_var_centering (int64_t group_id, 
                                const char * varname, 
                                const char * centering)
\end{lstlisting}

Input:
\begin{itemize}
\item group\_id - id of the internal group structure (returned by adios\_declare\_group call)
\item varname - string containing the variable name which is going to be visualized
\item centering - string containing the centering information of the variable (point or cell)
\end{itemize}

Fortran example:
\begin{lstlisting}[alsolanguage=Fortran,caption={},label={}]
call adios_define_var_centering (m_adios_group, "NX", "cell")
\end{lstlisting}


\subsection{adios\_define\_var\_timesteps}
This API defines the variable time steps. The timesteps point to time variables using steps, 
starting from step 0.


\begin{lstlisting}[alsolanguage=C,caption={},label={}]
int adios_define_var_timesteps (const char * timesteps, 
                                int64_t group_id, 
                                const char * name)
\end{lstlisting}

Input:
\begin{itemize}
\item timesteps - string containing time step of the variable on the mesh. There are three types of
timesteps. For more detailed information, please consult ADIOS-VisualizationSchema-1.1 manual,
which is available at https://users.nccs.gov/~pnorbert/ADIOS-VisualizationSchema-1.1.pdf
\item group\_id - id of the internal group structure (returned by adios\_declare\_group call)
\item name - string containing the variable name which is going to be visualized
\end{itemize}

Fortran example:
\begin{lstlisting}[alsolanguage=Fortran,caption={},label={}]
call adios_define_var_timesteps ("5", m_adios_group, "NX")
\end{lstlisting}


\subsection{adios\_define\_var\_timescale}
This API defines the variable time scale. The timescale points to time variables using real time, 
starting from time exactly the same as time steps except with real numbers. 

\begin{lstlisting}[alsolanguage=C,caption={},label={}]
int adios_define_var_timescale (const char * timescale, 
                                int64_t group_id, 
                                const char * name)
\end{lstlisting}

Input:
\begin{itemize}
\item timescale - string containing time scale of the variable on the mesh. There are three types of
timescales. For more detailed information, please consult ADIOS-VisualizationSchema-1.1 manual,
which is available at https://users.nccs.gov/~pnorbert/ADIOS-VisualizationSchema-1.1.pdf
\item group\_id - id of the internal group structure (returned by adios\_declare\_group call)
\item name - string containing the variable name which is going to be visualized
\end{itemize}

Fortran example:
\begin{lstlisting}[alsolanguage=Fortran,caption={},label={}]
call adios_define_var_timescale ("0,0.0015,200", m_adios_group, "NX")
\end{lstlisting}
                             

\subsection{adios\_define\_var\_timeseriesformat}
This API defines the variable time series format or padding pattern for images.
If this number is 4, then the time-steps for images will be padded with 0 up to 4 digit numbers.

\begin{lstlisting}[alsolanguage=C,caption={},label={}]
int adios_define_var_timeseriesformat (const char * timeseries, 
                                       int64_t group_id, 
                                       const char * name)
\end{lstlisting}

Input:
\begin{itemize}
\item timeseries - string containing time series format (integers)
\item group\_id - id of the internal group structure (returned by adios\_declare\_group call)
\item name - string containing the variable name which is going to be visualized
\end{itemize}

Fortran example:
\begin{lstlisting}[alsolanguage=Fortran,caption={},label={}]
call adios_define_var_timeseriesformat ("4", m_adios_group, "NX")
\end{lstlisting}


\subsection{adios\_define\_var\_hyperslab}
This API defines a variable hyper slab (sub set or super set). Use the concept of start, stride
and count in all dimensions of a variable to identify a subset of a dataset.

\begin{lstlisting}[alsolanguage=C,caption={},label={}]
int adios_define_var_hyperslab (const char * hyperslab, 
                                int64_t group_id, 
                                const char * name)
\end{lstlisting}

Input:
\begin{itemize}
\item hyperslab - string containing hyperslab (a number, a range or 3 numbers)
\item group\_id - id of the internal group structure (returned by adios\_declare\_group call)
\item name - string containing the variable name which is going to be visualized
\end{itemize}

Fortran example:
\begin{lstlisting}[alsolanguage=Fortran,caption={},label={}]
call adios_define_var_hyperslab ("0,32", m_adios_group, "NX")
\end{lstlisting}


\subsection{adios\_define\_mesh\_timevarying}
This API indicates the mesh will change with time or not.

\begin{lstlisting}[alsolanguage=C,caption={},label={}]
int adios_define_mesh_timevarying (const char * timevarying, 
                                   int64_t group_id, 
                                   const char * name)
\end{lstlisting}

Input:
\begin{itemize}
\item timevarying - string containing time varying information (yes or no)
\item group\_id - id of the internal group structure (returned by adios\_declare\_group call)
\item name - string containing the mesh name
\end{itemize}

Fortran example:
\begin{lstlisting}[alsolanguage=Fortran,caption={},label={}]
call adios_define_mesh_timevarying ("no", m_adios_group, "uniformmesh")
\end{lstlisting}


\subsection{adios\_define\_mesh\_timesteps}
This API define the time steps at the mesh level (for all variables on that mesh).

\begin{lstlisting}[alsolanguage=C,caption={},label={}]
int adios_define_mesh_timesteps (const char * timesteps, 
                                 int64_t group_id, 
                                 const char * name)
\end{lstlisting}

Input:
\begin{itemize}
\item timesteps - string containing time step of the mesh. There are three types of
timesteps. For more detailed information, please consult ADIOS-VisualizationSchema-1.1 manual,
which is available at https://users.nccs.gov/~pnorbert/ADIOS-VisualizationSchema-1.1.pdf
\item group\_id - id of the internal group structure (returned by adios\_declare\_group call)
\item name - string containing the mesh name
\end{itemize}

Fortran example:
\begin{lstlisting}[alsolanguage=Fortran,caption={},label={}]
call adios_define_mesh_timesteps ("1,32", m_adios_group, "uniformmesh")
\end{lstlisting}


\subsection{adios\_define\_mesh\_timescale}
This API define the time scale at the mesh level (for all variables on that mesh).

\begin{lstlisting}[alsolanguage=C,caption={},label={}]
int adios_define_mesh_timescale (const char * timescale, 
                                 int64_t group_id, 
                                 const char * name)
\end{lstlisting}

Input:
\begin{itemize}
\item timescale - string containing time scale of the mesh. There are three types of
time scales. For more detailed information, please consult ADIOS-VisualizationSchema-1.1 manual,
which is available at https://users.nccs.gov/~pnorbert/ADIOS-VisualizationSchema-1.1.pdf
\item group\_id - id of the internal group structure (returned by adios\_declare\_group call)
\item name - string containing the mesh name
\end{itemize}

Fortran example:
\begin{lstlisting}[alsolanguage=Fortran,caption={},label={}]
call adios_define_mesh_timescale ("0.1,3", m_adios_group, "uniformmesh")
\end{lstlisting}


\subsection{adios\_define\_mesh\_timeseriesformat}
This API define the time series formatting at the mesh level (for all variables on that mesh).

\begin{lstlisting}[alsolanguage=C,caption={},label={}]
int adios_define_mesh_timeseriesformat (const char * timeseries, 
                                        int64_t group_id, 
                                        const char * name)
\end{lstlisting}

Input:
\begin{itemize}
\item timeseries - string containing time series format (integers)
\item group\_id - id of the internal group structure (returned by adios\_declare\_group call)
\item name - string containing the mesh name
\end{itemize}

Fortran example:
\begin{lstlisting}[alsolanguage=Fortran,caption={},label={}]
call adios_define_mesh_timeseriesformat ("5", m_adios_group, "uniformmesh")
\end{lstlisting}


\subsection{adios\_define\_mesh\_group}
This API indicates where (which ADIOS group) mesh variables are stored.

\begin{lstlisting}[alsolanguage=C,caption={},label={}]
int adios_define_mesh_group (const char * group, 
                             int64_t group_id, 
                             const char * name)
\end{lstlisting}

Input:
\begin{itemize}
\item group - string containing ADIOS group name
\item group\_id - id of the internal group structure (returned by adios\_declare\_group call)
\item name - string containing the mesh name
\end{itemize}

Fortran example:
\begin{lstlisting}[alsolanguage=Fortran,caption={},label={}]
call adios_define_mesh_group ("experiment", m_adios_group, "uniformmesh")
\end{lstlisting}


\subsection{adios\_define\_mesh\_file}
This API indicates a external file where mesh is defined.

\begin{lstlisting}[alsolanguage=C,caption={},label={}]
int adios_define_mesh_file(int64_t group_id, char * name, char * file)
\end{lstlisting}

Input:
\begin{itemize}
\item group\_id - id of the internal group structure (returned by adios\_declare\_group call)
\item name - string containing the mesh name
\item file - string containing the file name storing the mesh definition
\end{itemize}

Fortran example:
\begin{lstlisting}[alsolanguage=Fortran,caption={},label={}]
call adios_define_mesh_file (m_adios_group, "uniformmesh", "uniformmesh.bp")
\end{lstlisting}


\subsection{adios\_define\_mesh\_uniform}
This function defines a uniform mesh. For not requried attributes in this function, please use 0 instead.

\begin{lstlisting}[alsolanguage=Fortran,caption={},label={}]
int adios_define_mesh_uniform (char * dimensions,
                               char * origin,
                               char * spacing,
                               char * maximum,
                               char * nspace,
                               int64_t group_id,
                               const char * name)
\end{lstlisting}

Input:
\begin{itemize}
\item dimensions - string containing the number of dimensions, required
\item origin - string containing the mesh origins (in all dimensions), not required (default 0)
\item spacing - string containing the mesh spacings (between points in all dimensions), not requried (default 1)
\item maximum - string containing the mesh maximums (in all dimensions), not required
\item nspace - string containing the number of spcaces of the mesh, not required (default is the number of dimension)
\item group\_id - id of the internal group structure (returned by adios\_declare\_group call), required
\item name - string containing mesh name, required
\end{itemize}

Fortran example:
\begin{lstlisting}[alsolanguage=Fortran,caption={},label={}]
call adios_define_mesh_uniform ("10,10,10", "0,1,0.5", "0.5, 0.3, 1", 0, 
                                "3", m_adios_group, "uniformmesh")
\end{lstlisting}


\subsection{adios\_define\_mesh\_rectilinear}
This function defines a rectilinear mesh. For not requried attributes in this function, please use 0 instead.

\begin{lstlisting}[alsolanguage=Fortran,caption={},label={}]
int adios_define_mesh_rectilinear (char * dimensions,
                                   char * coordinates,
                                   char * nspace,
                                   int64_t group_id,
                                   const char * name)
\end{lstlisting}

Input:
\begin{itemize}
\item dimensions - string containing the number of dimensions, required
\item coordinates - string containing the variable name(s) pointing to the mesh coordinates, required
\item nspace - string containing the number of spcaces of the mesh, not required (default is the number of dimension)
\item group\_id - id of the internal group structure (returned by adios\_declare\_group call), required
\item name - string containing mesh name, required
\end{itemize}

Fortran example:
\begin{lstlisting}[alsolanguage=Fortran,caption={},label={}]
call adios_define_mesh_rectilinear ("10,10,10", "X,Y", "2", m_adios_group, 
                                    "rectilinearmesh")
\end{lstlisting}


\subsection{adios\_define\_mesh\_structured}
This function defines a structured mesh. For not requried attributes in this function, please use 0 instead.

\begin{lstlisting}[alsolanguage=Fortran,caption={},label={}]
int adios_define_mesh_structured (char * dimensions,
                                  char * points,
                                  char * nspace,
                                  int64_t group_id,
                                  const char * name)
\end{lstlisting}

Input:
\begin{itemize}
\item dimensions - string containing the number of dimensions, required
\item points - string containing variable name(s) pointing to mesh points, required 
\item nspace - string containing the number of spcaces of the mesh, not required (default is the number of dimension)
\item group\_id - id of the internal group structure (returned by adios\_declare\_group call), required
\item name - string containing mesh name, required
\end{itemize}

Fortran example:
\begin{lstlisting}[alsolanguage=Fortran,caption={},label={}]
call adios_define_mesh_structured ("10,10,10", "X,Y", "2", m_adios_group, 
                                   "structuredmesh");
\end{lstlisting}


\subsection{adios\_define\_mesh\_unstructured}
This function defines a unstructured mesh. For not requried attributes in this function, please use 0 instead.

\begin{lstlisting}[alsolanguage=Fortran,caption={},label={}]
int adios_define_mesh_unstructured (char * points,
                                    char * data,
                                    char * count,
                                    char * cell_type,
                                    char * npoints,
                                    char * nspace,
                                    int64_t group_id,
                                    const char * name)
\end{lstlisting}

Input:
\begin{itemize}
\item points - string containing variable name(s) pointing to mesh points, required
\item data - string containing the variable name(s) pointing to the mesh cell data (arrays)
\item count - string containing numbers or variable names(s) pointing to the mesh cell counts, required
\item cell\_type - string containing cell types or variable names(s) pointing to the mesh cell types 
(line, triangle, quad, hex, prism, tet, pyr), required
\item nspace - string containing the number of spcaces of the mesh, not required (default is the number of dimension)
\item group\_id - id of the internal group structure (returned by adios\_declare\_group call), required
\item name - string containing mesh name, required
\end{itemize}

Fortran example:
\begin{lstlisting}[alsolanguage=Fortran,caption={},label={}]
call adios_define_mesh_unstructured ("points", "cells", "num_cells", 
                                     "triangle", 0, "2", m_adios_group, 
                                     "trimesh")
\end{lstlisting}
